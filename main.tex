\PassOptionsToPackage{table}{xcolor}
\documentclass{beamer}
\mode<presentation>
{
  \usetheme{Madrid}
  \usecolortheme{default}
  \usefonttheme{serif}
  \setbeamertemplate{navigation symbols}{}
  \setbeamertemplate{caption}[numbered]
} 

\usepackage[english]{babel}
\usepackage[utf8x]{inputenc}
\usepackage[version=3]{mhchem}
\usepackage{pgfpages}
\usepackage{amsmath}
\usepackage{csquotes}
\usepackage{soul}
\pgfpagesuselayout{resize to}[%
  physical paper width=8in, physical paper height=6in]
\title[Strassens Algorithmus]{Strassens Algorithmus \\ für die Matrixmultiplikation}
\author{Bashar Khoulani}
\institute[]{Universität Kassel}
\date{\today}

\begin{document}

\begin{frame}
    \titlepage
\end{frame}

\section{Einführung}
\begin{frame}{Einführung --- Anwendungen}
\uncover<+-> {
Matrixmultiplikation in der Anwendung:
\begin{itemize}[<+->]
\item Lösung von linearen Gleichungssystemen
\item Transitiven Abschluss eines Graphen berechnen
\item Inverse und Determinante einer Matrix
\item Kürzeste Pfade in einem Graphen finden
\end{itemize}
}
\end{frame}

\begin{frame}{Einführung --- Matrixmultiplikation}
\uncover<+->{
Matrixmultiplikation wie aus der linearen Algebra bekannt:
}
\uncover<+-> {
\[
\cdot: R^{n\times l} \times R^{l\times m} \rightarrow R^{n\times m}, (A, B) \mapsto C = A \cdot B
\]
\[\text{ mit } 
c_{ij} = \sum_{k=1}^{l} a_{ik}b_{kj} \text{, für }i = 1, \dots, n \text{ und }j = 1, \dots, m
\]
}
\uncover<+-> {
Beispiel: $\cdot: \mathbb{Z}^{n\times l} \times \mathbb{Z}^{l\times m} \rightarrow \mathbb{Z}^{n\times m}, (A, B) \mapsto C = A \cdot B$
\alt<3>{
\[
    \begin{pmatrix}
    \; 1 & 2 & 3 \text{ }\\
    \; 4 & 5 & 6 \text{ }\\
    \; 7 & 8 & 9 \text{ }
    \end{pmatrix} \cdot 
    \begin{pmatrix}
    \; 1 & 2 \text{ }\\
    \; 3 & 4 \text{ }\\
    \; 5 & 6 \text{ }
    \end{pmatrix} =
    \begin{pmatrix}
    \; \phantom{22} & \; \phantom{28} \text{ } \\
    \; \phantom{49} & \; \phantom{64} \text{ } \\
    \; \phantom{76} & \; \phantom{100} \text{ } 
    \end{pmatrix} 
\]}{
\alt<4>{
\[
    \left(\begin{array}{ccc}
    \rowcolor{red!20}
    1 & 2 & 3\\
    4 & 5 & 6\\
    7 & 8 & 9
    \end{array}\right) \cdot 
    \left(\begin{array}{>{\columncolor{blue!20}}cc}
    1 & 2\\
    3 & 4\\
    5 & 6
    \end{array}\right) =
    \left(\begin{array}{cccc}
    \cellcolor{green!10} 22 & \phantom{28} \\
    \phantom{49} & \phantom{64} \\
    \phantom{76} & \phantom{100} 
    \end{array}\right) 
\]}{
\alt<5>{
\[
    \left(\begin{array}{ccc}
    \rowcolor{red!20}
    1 & 2 & 3\\
    4 & 5 & 6\\
    7 & 8 & 9
    \end{array}\right) \cdot 
    \left(\begin{array}{c>{\columncolor{blue!20}}c}
    1 & 2\\
    3 & 4\\
    5 & 6
    \end{array}\right) =
    \left(\begin{array}{cccc}
    22 & \cellcolor{green!10} 28 \\
    \phantom{49} & \phantom{64} \\
    \phantom{76} & \phantom{100} 
    \end{array}\right) 
\]}{
\alt<6>{
\[
    \left(\begin{array}{ccc}
    1 & 2 & 3\\
    \rowcolor{red!20}
    4 & 5 & 6\\
    7 & 8 & 9
    \end{array}\right) \cdot 
    \left(\begin{array}{>{\columncolor{blue!20}}cc}
    1 & 2\\
    3 & 4\\
    5 & 6
    \end{array}\right) =
    \left(\begin{array}{cccc}
    22 & 28 \\
    \cellcolor{green!10} 49 & \phantom{64} \\
    \phantom{76} & \phantom{100} 
    \end{array}\right) 
\]
}{
\alt<7>{
\[
    \left(\begin{array}{ccc}
    1 & 2 & 3\\
    \rowcolor{red!20}
    4 & 5 & 6\\
    7 & 8 & 9
    \end{array}\right) \cdot 
    \left(\begin{array}{c>{\columncolor{blue!20}}c}
    1 & 2\\
    3 & 4\\
    5 & 6
    \end{array}\right) =
    \left(\begin{array}{cccc}
    22 & 28 \\
    49 & \cellcolor{green!10} 64 \\
    \phantom{76} & \phantom{100} 
    \end{array}\right) 
\]}{
\alt<8>{
\[
    \left(\begin{array}{ccc}
    1 & 2 & 3\\
    4 & 5 & 6\\
    \rowcolor{red!20}
    7 & 8 & 9
    \end{array}\right) \cdot 
    \left(\begin{array}{>{\columncolor{blue!20}}cc}
    1 & 2\\
    3 & 4\\
    5 & 6
    \end{array}\right) =
    \left(\begin{array}{cc}
    22 & 28 \\
    49 & 64 \\
    \cellcolor{green!10} 76 & \phantom{100} 
    \end{array}\right) 
\]}{
\[
    \left(\begin{array}{ccc}
    1 & 2 & 3\\
    4 & 5 & 6\\
    \rowcolor{red!20}
    7 & 8 & 9
    \end{array}\right) \cdot 
    \left(\begin{array}{c>{\columncolor{blue!20}}c}
    1 & 2\\
    3 & 4\\
    5 & 6
    \end{array}\right) =
    \left(\begin{array}{cc}
    22 & 28 \\
    49 & 64 \\
    76 & \cellcolor{green!10} 100 
    \end{array}\right) 
\]}}}}}}}

\pause[9]
\end{frame}

\begin{frame}{Einführung --- Algorithmen}
\uncover<+-> {
    In diesem Vortrag zu behandelnde Algorithmen:
    \begin{itemize}[<+->]
        \item \st{Der "naive" Algorithmus}
        \begin{itemize}
            \item[--] $\Theta(n^3)$
        \end{itemize}
        \item Divide-And-Conquer-Algorithmus
        \begin{itemize}
            \item[--] $\Theta(n^3)$
        \end{itemize}
        \item Strassens Algorithmus
        \begin{itemize}
            \item[--] $\Theta(n^{2.81})$
        \end{itemize}
    \end{itemize}
}
\end{frame}

\begin{frame}{Divide-And-Conquer}
    \uncover<+->{
        Annahme: $A$, $B$ und $C$ sind $2^k \times 2^k$-Matrizen \\
    }
    \uncover<+->{
        Partitioniere $A$, $B$ und $C$ folgendermaßen für $k \geq 1$:
        \begin{equation}
            A = \left(\begin{array}{cc}
                    A_{11} & A_{12} \\
                    A_{21} & A_{22} 
                \end{array}\right), \,
            B = \left(\begin{array}{cc}
                    B_{11} & B_{12} \\
                    B_{21} & B_{22} 
                \end{array}\right), \,
            C = \left(\begin{array}{cc}
                    C_{11} & C_{12} \\
                    C_{21} & C_{22} 
            \end{array}\right)
        \end{equation}
    }
    \uncover<+->{Dann gilt für das Produkt $A\cdot B = C$:
        \begin{equation}
            \left(\begin{array}{cc}
                A_{11} & A_{12} \\
                A_{21} & A_{22} 
            \end{array}\right) \cdot
            \left(\begin{array}{cc}
                B_{11} & B_{12} \\
                B_{21} & B_{22} 
            \end{array}\right) =
            \left(\begin{array}{cc}
                C_{11} & C_{12} \\
                C_{21} & C_{22} 
            \end{array}\right)
        \end{equation}
    }
    \uncover<+->{Mit
    \begin{equation}
        C_{11} = A_{11} \cdot B_{11} + A_{12} \cdot B_{21}
    \end{equation}
    \begin{equation}
        C_{12} = A_{11} \cdot B_{12} + A_{12} \cdot B_{22}
    \end{equation}
    \begin{equation}
        C_{21} = A_{21} \cdot B_{11} + A_{22} \cdot B_{21}
    \end{equation}
    \begin{equation}
        C_{22} = A_{21} \cdot B_{12} + A_{22} \cdot B_{22}
    \end{equation}
    }
\end{frame}


\section{Strassens Algorithmus}
\begin{frame}{Strassens Algorithmus}
    
\end{frame}

\section{Bewertung}
\begin{frame}{Bewertung}
    
\end{frame}

\section{Ausblick}
\begin{frame}{Ausblick}
    
\end{frame}

\section{Literatur}
\begin{frame}{Literatur}
    \nocite{*}
    \bibliographystyle{plain}
    \bibliography{sem.bib}
\end{frame}

\end{document}